\documentclass{article}
\usepackage{listings}
\usepackage{svg}
\usepackage{adjustbox}
\usepackage[utf8]{inputenc}
\usepackage[T1]{fontenc}
\usepackage{enumitem}
\usepackage{geometry}
\usepackage{float}
\usepackage[section]{placeins}
\geometry{a4paper,total={170mm,257mm},left=20mm,top=20mm,}
\title{Sprawozdanie z efektywności wskaźnika giełdowego MACD}
\author{Anna Berkowska}
\date{16 marca 2024}
\begin{document}
    \maketitle
    \section*{Wstęp}

    Wskaźnik MACD (Moving Average Convergence Divergence) jest jednym z najbardziej popularnych narzędzi analizy technicznej używanych do identyfikacji trendów oraz sygnałów kupna i sprzedaży na rynku finansowym. Pomimo swojej popularności, MACD, podobnie jak wiele innych wskaźników technicznych, posiada pewne ograniczenia, w tym opóźnienie w generowaniu sygnałów.

    Wskaźnik ten składa się z dwóch wykresóœ: MACD i linii sygnału (SIGNAL). Miejsce, w którym MACD przecina SIGNAL od dołu jest sygnałem do zakupu akcji. Miejsce, w którym MACD przecina SIGNAL od góry, jest sygnałem do sprzedaży akcji.

    Celem niniejszego projektu jest symulacja strategii kupna i sprzedaży kryptowaluty ETHUSDC przy użyciu wskaźnika MACD. W projekcie zostanie przedstawiony kod programu, który umożliwia automatyczną symulację działania strategii opartej na sygnałach MACD. Zbadane zostaną również wyniki tej strategii w kontekście potencjalnych zysków i strat.

    \section*{Opis przebiegu symulacji}

    W ramach przeprowadzonej symulacji, zbadano skuteczność strategii inwestycyjnej opartej na wskaźniku MACD w kontekście handlu kryptowalutami ETHUSDC. W ramach symulacji, zbadano dwa scenariusze inwestycyjne:

    \begin{enumerate}
        \item Inwestycja początkowa w wysokości \$2,252,000.98 (1000 ETHUSDC), z użyciem strategii opartej na wskaźniku MACD (algorytm podany został przedstawiony poniżej).
        \item Inwestycja początkowa w wysokości \$2,252,000.98 (1000 ETHUSDC), z użyciem strategii prostego zakupu i sprzedaży wszystkich monet na początku i na końcu okresu analizy.
    \end{enumerate}

    Pod uwagę wzięto okres analizy wynoszący 1000 godzin. Wszelkie obliczenia wykonywane były na podstawie godziny zamknięcia. W ramach symulacji, zbadano wartość portfela inwestycyjnego w zależności od zastosowanej strategii inwestycyjnej.

    W przypadku strategii opartej na wskaźniku MACD, wartość portfela inwestycyjnego na koniec analizowanego okresu wynosiła \$2,729,377.71, co jest równowartością 121.20\% wartości początkowej.

    W przypadku strategii prostego zakupu i sprzedaży, wartość portfela inwestycyjnego na koniec analizowanego okresu wynosiła \$3,890,858.51, co jest równowartością 172.77\% wartości początkowej.

    Różnica w wartości portfela inwestycyjnego pomiędzy zastosowanymi strategiami wynosi \$1,161,480.80, co stanowi różnicę w wysokości 51.57 punktów procentowych.

    W ramach symulacji wygenerowane zostały wykresy cen kryptowaluty ETHUSDC, MACD wraz z linią sygnałową, a także wykres wartości portfela w ciągu całej symulacji. W celu lepszego zrozumienia wyników, wygenerowane zostały także wykresy na podstawie ostatnich 200 godzin z okresu analizy.

    \begin{figure}[H]
        \begin{lstlisting}[language=Python, caption={Source code in Python}, label={lst:lstlisting}, frame=single, numbers=left]
    def investing_algorithm(initial_funds, prices, macd_values, signal_values):
        funds = initial_funds
        coins = 0
        portfolio_values = [initial_funds]
        buy_points = []
        sell_points = []

        for i in range(1, len(macd_values)):
            if any(x is None for x in [macd_values[i], macd_values[i - 1],
                                       signal_values[i], signal_values[i - 1]]):
                portfolio_values.append(funds + coins * prices[i])
                continue
            if macd_values[i] > signal_values[i] and \
               macd_values[i - 1] <= signal_values[i - 1]:
                # Buy signal
                if funds > 0:
                    buy_points.append(i)
                    coins_to_buy = funds / prices[i]
                    coins += coins_to_buy
                    funds -= coins_to_buy * prices[i]
            elif macd_values[i] < signal_values[i] and \
                 macd_values[i - 1] >= signal_values[i - 1]:
                # Sell signal
                if coins > 0:
                    sell_points.append(i)
                    funds += coins * prices[i]
                    coins = 0
            portfolio_values.append(funds + coins * prices[i])

        final_funds = funds + coins * prices.iloc[-1]
        return final_funds, portfolio_values, buy_points, sell_points


    def simple_algorithm(initial_funds, prices):
        funds = initial_funds
        coins = 0

        # Buy all the coins at the beginning
        if funds > 0:
            coins_to_buy = funds / prices[0]
            coins += coins_to_buy
            funds -= coins_to_buy * prices[0]

        final_funds = funds + coins * prices.iloc[-1]
        return final_funds
        \end{lstlisting}\label{fig:figure}
    \end{figure}

    \section*{Wykresy i wyniki}

    \begin{figure}[H]
        \centering
        \makebox[\textwidth][c]{\includesvg[width=1.4\textwidth]{/home/anna/University/Projects/fourth-semester/MN/macd/plots/ETHUSDC/close_price_plot_1000.svg}}
        \caption{Close Price Plot}
        \label{fig:close_price_1000}
    \end{figure}

    \begin{figure}[H]
        \centering
        \makebox[\textwidth][c]{\includesvg[width=1.4\textwidth]{/home/anna/University/Projects/fourth-semester/MN/macd/plots/ETHUSDC/macd_plot_1000.svg}}
        \caption{MACD Plot}
        \label{fig:macd_plot_1000}
    \end{figure}

    \begin{figure}[H]
        \centering
        \makebox[\textwidth][c]{\includesvg[width=1.4\textwidth]{/home/anna/University/Projects/fourth-semester/MN/macd/plots/ETHUSDC/portfolio_plot_1000.svg}}
        \caption{Portfolio Plot}
        \label{fig:portfolio_plot_1000}
    \end{figure}

     \begin{figure}[H]
        \centering
        \makebox[\textwidth][c]{\includesvg[width=1.4\textwidth]{/home/anna/University/Projects/fourth-semester/MN/macd/plots/ETHUSDC/close_price_plot_200.svg}}
        \caption{Close Price Plot}
        \label{fig:close_price_200}
    \end{figure}

    \begin{figure}[H]
        \centering
        \makebox[\textwidth][c]{\includesvg[width=1.4\textwidth]{/home/anna/University/Projects/fourth-semester/MN/macd/plots/ETHUSDC/macd_plot_200.svg}}
        \caption{MACD Plot}
        \label{fig:macd_plot_200}
    \end{figure}

    \begin{figure}[H]
        \centering
        \makebox[\textwidth][c]{\includesvg[width=1.4\textwidth]{/home/anna/University/Projects/fourth-semester/MN/macd/plots/ETHUSDC/portfolio_plot_200.svg}}
        \caption{Portfolio Plot}
        \label{fig:portfolio_plot_200}
    \end{figure}

    \section*{Zakończenie}

    Na podstawie przedstawionej symulacji, można stwierdzić, że strategia inwestycyjna oparta na wskaźniku MACD nie jest w stanie zapewnić takich samych zysków, jak strategia prostego zakupu i sprzedaży. Wartość portfela inwestycyjnego na koniec analizowanego okresu była o 51.57 punktów procentowych niższa w przypadku strategii opartej na wskaźniku MACD w porównaniu do strategii prostego zakupu i sprzedaży.

    Wynika to z opóźnienia w generowaniu sygnałów przez wskaźnik MACD, co sprawia, że inwestorzy mogą przegapić okazje do zakupu i sprzedaży. Jednakże, mimo niższej efektywności, analiza wskaźnika MACD może nadal być użyteczna jako jedno z wielu narzędzi w analizie technicznej, umożliwiając inwestorom lepsze zrozumienie dynamiki rynku i potencjalnych kierunków zmian cen kryptowalut.

    Ważne jest, aby inwestorzy zrozumieli ograniczenia wskaźnika MACD i rozważyli jego zastosowanie w połączeniu z innymi narzędziami i strategiami analizy technicznej w celu zwiększenia efektywności inwestycji i minimalizacji ryzyka. Dalsze badania i testowanie różnych strategii handlowych mogą przyczynić się do lepszego zrozumienia zachowania wskaźnika MACD oraz jego potencjalnego zastosowania w praktyce inwestycyjnej na rynku kryptowalut.
\end{document}